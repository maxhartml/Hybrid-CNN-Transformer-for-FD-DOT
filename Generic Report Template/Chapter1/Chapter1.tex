% !TEX root =  ../Dissertation.tex

\chapter{Introduction}

\section{Background and Motivation}

% >>> STARTER: Background and Motivation
DOT uses NIR light to infer tissue absorption ($\mu_a$) and reduced scattering ($\mu'_s$), but mapping sparse boundary measurements to 3D volumes is severely ill-posed. This work motivates a modern, data-driven approach that retains physical plausibility while achieving fast, reliable reconstructions suitable for clinical workflows. We position the problem within quantitative NIR-DOT and argue for architectures that learn robust, geometry-aware representations.
% <<< STARTER: Background and Motivation

\section{Near-Infrared Diffuse Optical Tomography (NIR-DOT)}

% >>> STARTER: Near-Infrared Diffuse Optical Tomography (NIR-DOT)
We briefly review photon migration in highly scattering media and the diffusion approximation used in practice for NIR-DOT. The forward model maps tissue optical properties to detector measurements (amplitude and phase), whereas the inverse task estimates volumetric $\mu_a$ and $\mu'_s$ from those measurements. The modality offers non-ionising, cost-effective, functional contrast compared with MRI/CT.
% <<< STARTER: Near-Infrared Diffuse Optical Tomography (NIR-DOT)

\section{Challenges in NIR-DOT Reconstruction}

% >>> STARTER: Challenges in NIR-DOT Reconstruction
The inverse problem is underdetermined, unstable to noise, and sensitive to modelling mismatches and probe geometry. Limited source–detector pairs and exponential attenuation exacerbate ambiguity when reconstructing $64^3$ volumes. Robust solutions must integrate data-driven learning with constraints consistent with photon transport physics.
% <<< STARTER: Challenges in NIR-DOT Reconstruction

\section{Research Objectives and Contributions}

% >>> STARTER: Research Objectives and Contributions
We propose a two-stage CNN–Transformer pipeline with latent alignment that reconstructs 3D optical property maps from frequency-domain measurements. Key contributions include: a physics-aware synthetic data pipeline, a 3D CNN autoencoder for spatial features, a transformer encoder for measurement sequences with spatial embeddings, and a teacher–student latent alignment objective enabling decoder reuse. We evaluate in raw units with clinically meaningful metrics.
% <<< STARTER: Research Objectives and Contributions

\section{Dissertation Structure}

% >>> STARTER: Dissertation Structure
The thesis proceeds from background and related work to data generation, model architecture, training methodology, and results, before discussing limitations and future directions. Appendices compile implementation details, additional results, and mathematical derivations. This mirrors the experimental pipeline from simulation to evaluation.
% <<< STARTER: Dissertation Structure