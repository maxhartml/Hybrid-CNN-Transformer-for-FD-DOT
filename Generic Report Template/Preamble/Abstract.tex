
\chapter*{Abstract}
\addcontentsline{toc}{chapter}{Abstract}

This dissertation presents a novel hybrid CNN-Transformer architecture for enhanced Near-Infrared Diffuse Optical Tomography (NIR-DOT) reconstruction. NIR-DOT is a non-invasive imaging technique with significant potential for brain imaging and cancer detection, but faces challenges in reconstruction accuracy due to the ill-posed nature of the inverse problem. 

We introduce a two-stage approach combining the spatial feature learning capabilities of CNNs with the contextual processing power of Transformers, further enhanced by the integration of local tissue context information. Using a comprehensive dataset of 5,000 synthetic phantoms, we demonstrate that our approach significantly improves reconstruction quality over traditional methods and standard deep learning approaches. 

Our key innovation—tissue context integration—provides physiologically relevant priors that guide the reconstruction process. Experimental results show improvements in RMSE, SSIM, and PSNR metrics, with particularly notable enhancements in tumor region reconstruction. The proposed method represents a significant advance in NIR-DOT reconstruction technology with potential clinical applications in brain imaging and breast cancer detection.