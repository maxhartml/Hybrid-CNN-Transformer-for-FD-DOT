\chapter{Introduction}

\section{Background and Motivation}
Diffuse optical tomography (DOT) is a non-invasive imaging modality that reconstructs the optical properties of tissue from near-infrared (NIR) light measurements. By exploiting the wavelength-dependent absorption and scattering of NIR photons in biological tissue, DOT enables three-dimensional imaging of physiological parameters such as blood volume and oxygenation. These parameters are biomarkers of vascularisation and haemodynamics, making DOT attractive in oncology and neuroscience. Unlike ionising modalities such as CT or PET, DOT is safe for repeated use and portable. These features make it well suited to longitudinal monitoring, point-of-care screening, and intraoperative guidance \cite{arridge1999, gibson2005}.

These advantages translate directly into clinical impact. Breast cancer is the most common cancer in women, where early detection and treatment monitoring are critical for outcomes. DOT provides functional information such as tumour oxygenation and haemoglobin concentration, not readily accessible with conventional imaging. In particular, DOT has been investigated for monitoring patient response to neoadjuvant chemotherapy, where frequent, non-invasive, and low-cost imaging is required but impractical with MRI or mammography \cite{tromberg2016}. In neuroscience, DOT-based functional imaging has been applied to study brain activation and cerebral oxygenation, offering a portable, low-cost alternative to fMRI, valuable at the bedside or in paediatrics \cite{eggebrecht2014}.

Recent advances in instrumentation have further expanded the potential of DOT. Handheld and wearable systems are now capable of real-time scanning and adapting to varied patient geometries \cite{stillwell2022}. These developments shift the bottleneck to computation: reconstruction must be rapid and robust to probe variability, anatomy, and noise. Conventional algorithms based on iterative inversion of the diffusion equation with repeated Jacobian updates remain prohibitively slow, often requiring minutes per reconstruction even on high-performance hardware. Deep learning-based DOT (DL-DOT) has therefore emerged as a promising paradigm, offering sub-second inference while maintaining or improving upon the fidelity of physics-based solvers \cite{dale2024}.

\section{Frequency-Domain Diffuse Optical Tomography}
DOT relies on the propagation of near-infrared (NIR) light, typically within the $650$–$900$\,nm optical window. This range maximises haemoglobin contrast while minimising water and lipid absorption, enabling centimetre-scale penetration. The key optical parameters are the absorption coefficient $\mu_a$ (mm$^{-1}$), reflecting chromophore concentration, and the reduced scattering coefficient $\mu'_s$ (mm$^{-1}$), influenced by tissue microstructure. Together, these parameters govern photon fluence and are the quantities to be reconstructed \cite{arridge1999, gibson2005}.

Measurements use arrays of sources and detectors on the tissue surface. Each source–detector (SD) pair samples a diffuse photon path, with separation (SDS) strongly influencing depth sensitivity. In frequency-domain DOT (FD-DOT), the light is sinusoidally modulated at radiofrequency (e.g.\ 140\,MHz); the detected signal is described by amplitude attenuation and phase shift relative to the input. Using $\log A$ (log-amplitude) together with $\phi$ (phase) provides complementary sensitivity to $\mu_a$ and $\mu'_s$. Short SDS ($<15$\,mm) probe superficial layers, whereas larger separations ($30$–$40$\,mm) reach deeper tissue at reduced SNR.

The forward model follows from the frequency-domain diffusion equation and is solved for realistic geometries using the finite element method (FEM) \cite{dehghani2009}. The operator $\mathcal{F}$ maps spatial fields of $\mu_a$ and $\mu'_s$ to boundary measurements. Recovering these parameters from sparse, surface-only data is underdetermined and therefore requires regularisation—via smoothness constraints, sparsity-promoting penalties, or data-driven priors learned by neural networks \cite{arridge1999}.

\textbf{Terminology.} Hereafter, \emph{FD-DOT} denotes frequency-domain measurements of amplitude and phase; \emph{DOT} the modality in general; \emph{DL-DOT} deep learning-based reconstruction.

\section{Problem Formulation and Notation}
This dissertation focuses on frequency-domain diffuse optical tomography (FD-DOT), where the photon field is sinusoidally modulated at frequency $f$ (Hz). For each source–detector (SD) pair, the measurement is expressed as a complex value:
\[
M = A e^{i\phi},
\]
where $A$ is the detected amplitude and $\phi$ is the phase shift relative to the source. Because amplitudes span several orders of magnitude, reconstructions use the logarithm of amplitude, $\log A$, together with $\phi$. These two quantities form the core measurement features for each SD pair.

Each SD pair is further represented by the three-dimensional coordinates of both the source and detector positions, $(x_s, y_s, z_s)$ and $(x_d, y_d, z_d)$. The complete feature vector for a single pair is therefore:
\[
m_i = \{ \log A_i, \phi_i, x_{s,i}, y_{s,i}, z_{s,i}, x_{d,i}, y_{d,i}, z_{d,i} \},
\]
integrating optical and spatial context. For a scan comprising $N$ SD pairs, the full measurement tensor is:
\[
\mathbf{y} = \{ m_1, m_2, \dots, m_N \} \in \mathbb{R}^{N \times 8}.
\]

The forward model of FD-DOT is governed by the frequency-domain diffusion equation, which for realistic geometries is solved numerically using the finite element method (FEM). This defines the mapping:
\[
\mathbf{y} = \mathcal{F}(\mu_a, \mu'_s) + \epsilon,
\]
where $\mathcal{F}$ denotes the FEM-based forward operator from optical properties to boundary measurements, and $\epsilon$ denotes additive noise, modelled as Gaussian perturbations with 0.5\% relative variance on $\log A$ and $\pm 0.5^\circ$ absolute variance on $\phi$. This simplified model captures FD-DOT sensitivity to amplitude and phase fluctuations while remaining tractable.

The task is to estimate voxelwise distributions of $\mu_a$ and $\mu'_s$ on a three-dimensional grid. This study adopts a $64 \times 64 \times 64$ discretisation at 1\,mm resolution, yielding approximately $2.6 \times 10^5$ voxels per parameter, or about $5.2 \times 10^5$ unknowns in total. Denoting the reconstructions by $\hat{\mu}_a$ and $\hat{\mu}'_s$, the inverse mapping can be expressed as:
\[
\mathcal{G}: \mathbf{y} \mapsto \{ \hat{\mu}_a, \hat{\mu}'_s \},
\]
where $\mathcal{G}$ is implemented by a learned neural network.

This inverse problem is severely underdetermined. Even with $N=1000$ SD pairs, the measurement tensor $\mathbf{y} \in \mathbb{R}^{1000 \times 8}$ contains far fewer entries than the hundreds of thousands of voxel values to be recovered. Moreover, measurement sensitivity is non-uniform, with superficial voxels contributing more than deeper ones. This imbalance renders the problem intrinsically unstable in the absence of strong priors. The central challenge is designing models and training strategies that embed spatial priors, handle probe geometry variability, and generalise across diverse phantoms while maintaining fidelity in both $\mu_a$ and $\mu'_s$.

\section{Challenges in FD-DOT Reconstruction}
The challenges in FD-DOT arise from both the physics of light transport and the requirements of clinical deployment:

\begin{itemize}
    \item \textbf{Ill-posedness:} Sparse, surface-only measurements must be mapped to dense three-dimensional volumes. Each detector records photons that have undergone multiple scattering events, producing overlapping sensitivity profiles. Deep tissue contributes weak signals, amplifying inversion instability. Without strong priors, reconstructions overfit superficial structures while failing to capture deeper inclusions.
    
    \item \textbf{Geometry shift:} In handheld or wearable systems, probe geometry is variable. Source–detector separations vary with operator handling, patient anatomy, and motion. FEM-based solvers can accommodate arbitrary layouts by recomputing Jacobians, but most deep learning models are trained on fixed geometries and degrade when layouts change \cite{dale2025}. Overcoming this limitation is essential for achieving path-agnostic and clinically practical FD-DOT.
    
    \item \textbf{Noise robustness:} FD-DOT measurements are influenced by electronic noise, coupling variability, and instrumental fluctuations. In this work, noise is modelled as Gaussian perturbations (0.5\% on $\log A$ and $\pm 0.5^\circ$ on $\phi$), a simplification reflecting typical variability. Even small perturbations can destabilise reconstructions unless considered during training, underscoring the need for noise-aware pipelines.
    
    \item \textbf{Sim-to-real gap:} Large-scale synthetic datasets enable supervised training but cannot fully replicate the heterogeneity of patient anatomy, motion artefacts, or hardware imperfections. This mismatch introduces a domain gap between simulated and clinical data, which must be narrowed for reliable DL-DOT deployment.
    
    \item \textbf{Latency:} Real-time use requires reconstructions in under 0.1\,s per volume. Iterative solvers are too slow, demanding hundreds of iterations per scan. Learned inverse solvers are therefore essential to achieve clinically viable runtimes while maintaining image quality.
\end{itemize}

\section{Research Objectives and Contributions}
The overarching aim of this dissertation is to advance diffuse optical tomography (DOT) towards models that generalise across diverse probe geometries and anatomies, addressing a central limitation of existing deep learning-based DOT (DL-DOT) approaches. Building on the hybrid CNN–Transformer paradigm introduced by Dale \cite{dale2024, dale2025}, this work investigates new strategies in data generation, architectural refinement, and evaluation to improve robustness and clinical viability. The principal objectives and contributions are as follows:

\begin{enumerate}
    \item \textbf{Phantom and probe diversity for generalisation:} A high-throughput phantom generation pipeline was developed that extends well beyond Dale’s slab-like tissue models. Ellipsoidal tissue volumes were embedded inside cubic air domains, creating tissue–air boundaries from which local surface patches defined source–detector placement. Tumour inclusions varied in size and shape, while probe positions were randomly distributed across accessible surfaces. To avoid spatial bias, phantoms were randomly rotated in three dimensions using the full SO(3) rotation group.
    
    \item \textbf{Systematic geometry randomisation:} Each phantom produced 1000 source–detector (SD) measurements, dynamically subsampled into fixed 256-token sequences. This enforces invariance to probe placement, provides strong data augmentation, and addresses degradation under geometry shift—one of the key barriers to clinical deployment of DL-DOT.
    
    \item \textbf{Hybrid CNN–Transformer framework:} A two-stage hybrid network was implemented in line with Dale’s design philosophy, but developed independently without access to specific architectural details. Stage~1 trains a 3D CNN autoencoder on ground-truth absorption and scattering volumes to establish a spatial latent representation. Stage~2 employs a transformer encoder with spatially aware embeddings that fuse optical measurements ($\log A$, $\phi$) with explicit source–detector coordinates, enabling robust volumetric reconstruction.
    
    \item \textbf{Architectural refinements:} Beyond the baseline design, improvements include an enhanced spatial embedding scheme, multi-query attention for global aggregation, and selective fine-tuning of the CNN decoder. Additional refinements are described in Chapter~4. Together, these modifications increase the expressivity and stability of the learned solver.
    
    \item \textbf{Evaluation of robustness:} The framework was assessed on held-out phantoms from the synthetic test set, providing unseen anatomical shapes and probe configurations. Comparative reference to Dale’s baseline work contextualises the results. The aim was not to surpass prior performance but to investigate whether the proposed strategies improved generalisation under phantom and probe variability.
\end{enumerate}

In summary, this dissertation explores how phantom diversity, geometry randomisation, and architectural refinements can be combined within a hybrid CNN–Transformer framework to investigate pathways towards more generalisable DL-DOT. The focus is on testing robustness to geometry variation, phantom diversity, and measurement noise, with the broader goal of informing future work on clinically viable reconstruction methods.