% !TEX root =  ../Dissertation.tex

\chapter{Discussion}

\section{Key Findings and Insights}

% >>> STARTER: Key Findings and Insights
The hybrid pipeline achieves coherent 3D reconstructions with a compact latent space, and Stage 2 consistently outperforms Stage 1. Spatially-aware tokenisation is central to geometry generalisation.
% <<< STARTER: Key Findings and Insights

% Discussion of key findings from the research

\section{Clinical Implications of Generalisable DOT Models}

% >>> STARTER: Clinical Implications of Generalisable DOT Models
Path-agnostic inference reduces per-device retraining costs and better matches handheld/bedside scenarios. Real-time potential arises from a single forward pass once trained.
% <<< STARTER: Clinical Implications of Generalisable DOT Models

% Analysis of clinical implications and potential applications

\section{Limitations of Current Approach}

% >>> STARTER: Limitations of Current Approach
Training uses only synthetic data; $\mu'_s$ remains harder to recover sharply; memory footprint is non-trivial for $64^3$ volumes. Generalisation to real hardware requires calibration and domain adaptation.
% <<< STARTER: Limitations of Current Approach

% Honest assessment of limitations and constraints (synthetic-only, memory cost, measurement density)

\section{Computational Efficiency Considerations}

% >>> STARTER: Computational Efficiency Considerations
Notes throughput, mixed-precision benefits, and the cost of forward solves during data generation. Outlines profiling hotspots and possible compression strategies.
% <<< STARTER: Computational Efficiency Considerations

% Analysis of computational requirements and efficiency considerations

