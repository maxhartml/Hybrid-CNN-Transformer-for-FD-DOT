% !TEX root =  ../Dissertation.tex

\chapter{Synthetic Phantom Data Generation}

\section{Physics-Based Forward Modelling}

% >>> STARTER: Physics-Based Forward Modelling
Describes frequency-domain diffusion forward solves (e.g., 140 MHz) producing amplitude and phase per source–detector pair. Notes tetrahedral meshing and unit conventions. Establishes the mapping from $\mu_a$, $\mu'_s$ volumes to measurements.
% <<< STARTER: Physics-Based Forward Modelling

% Description of the forward modeling approach using finite element methods

\section{Geometric Phantom Construction}

% >>> STARTER: Geometric Phantom Construction
Details $64 \times 64 \times 64$ voxel volumes with ellipsoidal tissues and randomly placed tumour inclusions. Random rotations and shape parameters reduce bias and encourage invariance. Controls ensure inclusions are well within tissue boundaries.
% <<< STARTER: Geometric Phantom Construction

% Methodology for creating realistic geometric phantoms

\section{Optical Property Assignment}

% >>> STARTER: Optical Property Assignment
Specifies physiologically plausible ranges for $\mu_a$ and $\mu'_s$ and scaling for tumours relative to background. Clarifies refractive index assumptions if fixed. Ensures values remain within clinical bounds during synthesis.
% <<< STARTER: Optical Property Assignment

% Process for assigning optical properties to different tissue types

\section{Surface Extraction and Probe Placement}

% >>> STARTER: Surface Extraction and Probe Placement
Explains surface-aware placement of ~50 sources and ~20 detectors per source with SDS in 10–40 mm. Encodes each measurement with [log-amplitude, phase, src\_xyz, det\_xyz]. Clarifies pairing logic and avoidance of overlaps.
% <<< STARTER: Surface Extraction and Probe Placement

% Details on surface mesh extraction and probe positioning

\section{Noise Model}

% >>> STARTER: Noise Model
States additive noise levels for amplitude/phase (e.g., relative Gaussian on log-amp; small absolute Gaussian on phase). Justifies where noise is injected (pre/post transforms) and why these levels are realistic. Notes any fixed seeds for repeatability.
% <<< STARTER: Noise Model

% Description of the noise model used to simulate measurement noise

\section{Dataset Composition and Preprocessing}

% >>> STARTER: Dataset Composition and Preprocessing
Summarises dataset size, splits, and standardisation (per-channel z-score for volumes and features). Undersampling: always select exactly 256 from 1000 measurements per phantom. Mentions file formats and loader behaviour.
% <<< STARTER: Dataset Composition and Preprocessing

% Overview of the generated dataset with 10,000 phantoms
